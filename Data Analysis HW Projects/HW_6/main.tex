\documentclass{report}
\setlength{\parindent}{0pt} % Sets the automatic indent size
\usepackage{graphicx} % Required for inserting images
\usepackage{pictex}   % Ensure PicTeX is available
\usepackage{pgfplots}
\usepackage{filecontents}
\usepackage{amsmath}
\usepackage{amssymb}
\usepackage{listings}
\usepackage{xcolor}
\usepackage{float}
\usepackage{booktabs}


\lstset{
    language=C++,
    basicstyle=\ttfamily\tiny,
    keywordstyle=\color{blue}\bfseries,
    commentstyle=\color{gray}\itshape,
    stringstyle=\color{red},
    numbers=none,
    numberstyle=\tiny,
    stepnumber=1,
    breaklines=true,
    frame=single,
    captionpos=b,
    showstringspaces=false
}

\title{HW\#6}
\author{
Elhaam Bhuiyan,
Amir Samarxhiu,
Shaqib Syed
}
\date{\today}

\begin{document}

\maketitle

\section*{Introduction}

In this project, we apply the Metropolis algorithm to the portfolio optimization problem, where the goal is to identify portfolio allocations that minimize variance under various constraints. We consider portfolios comprising the 30 stocks in the Dow Jones Industrial Average (DJIA), and compute portfolio variance using the empirical covariance matrix $V \in \mathbb{R}^{30 \times 30}$ of daily changes in value observed over a one-year period. \\

The Metropolis algorithm is a probabilistic method that searches for low-energy solutions by proposing random changes to a current state and deciding whether to accept them based on the change in the energy function. Transitions to lower energy states are accepted with maximum probability, while moves to higher energy states are accepted with a probability that decreases exponentially with the increase in energy, controlled by a temperature parameter $T$. This enables the algorithm to occasionally explore higher energy states to avoid getting stuck in local minima. \\

In our case, we define the energy of a portfolio $\mathbf{x} \in \mathbb{R}^{30}$, where $\sum x_i = 100$, as the variance of its daily change. More precisely,
\[
E(\mathbf{x}) = \mathbf{x}^TV\mathbf{x}
\]

Using the Metropolis algorithm and our chosen energy function, we explore a discrete state space of portfolios, each satisfying a different set of constraints. These restrictions include short-selling limits, upper and lower bounds on allocations, and structured templates such as "silly" portfolios that follow fixed allocation patterns. By simulating this constrained Markov chain, we can identify low-variance portfolios that comply with the given rules and approximate the minimum-variance solution within each setting. 

\pagebreak

\section*{Results:}

\subsection*{Problem 1:}

We aim to show that the variance of a portfolio 
\[
\mathbf{p} = \lambda \mathbf{x} + (1-\lambda) \mathbf{y}
\]
where \(\mathbf{x} \neq \mathbf{y}\), is a quadratic function in \(\lambda\). Since portfolio variance is given by $\text{Var}(\mathbf{z})=\mathbf{z}^TV\mathbf{z}$, we substitute $\mathbf{p}$ and expand:
\begin{align*}
\text{Var}(\mathbf{p}) 
&= (\lambda \mathbf{x} + (1 - \lambda)\mathbf{y})^T V (\lambda \mathbf{x} + (1 - \lambda)\mathbf{y}) \\
&= \lambda^2 \mathbf{x}^T V \mathbf{x} + 2\lambda(1 - \lambda)\mathbf{x}^T V \mathbf{y} + (1 - \lambda)^2 \mathbf{y}^T V \mathbf{y} \\
&= \left( \mathbf{x}^T V \mathbf{x} - 2\mathbf{x}^T V \mathbf{y} + \mathbf{y}^T V \mathbf{y} \right)\lambda^2 
  + \left( 2\mathbf{x}^T V \mathbf{y} - 2\mathbf{y}^T V \mathbf{y} \right)\lambda 
  + \mathbf{y}^T V \mathbf{y}
\end{align*}

It is clear that this expression is quadratic in $\lambda$, and can be written in the standard form
\[
\text{Var}(\mathbf{p}) = a\lambda^2 + b\lambda+c,
\]
where 
\[
a= \mathbf{x}^T V \mathbf{x} - 2\mathbf{x}^T V \mathbf{y} + \mathbf{y}^T V \mathbf{y} = (\mathbf{x} - \mathbf{y})^TV(\mathbf{x} - \mathbf{y}) > 0,
\]
since $V$ is positive definite and $\mathbf{x} \neq \mathbf{y}$. Therefore, $\text{Var}(\mathbf{p})$ is a strictly convex function of $\lambda$, and has a unique global minimum. \\

This implies that the portfolio variance function has no local minima besides the global one. In unconstrained settings such as Problem 2, this means the Metropolis algorithm does not risk becoming trapped in suboptimal states. A low $T$ or $T = 0$ is sufficient to find the minimum variance portfolio efficiently. \\

The same reasoning applies to Problem 3 because short positions are not allowed, meaning each component $x_i$ must be non-negative. The set of allowed portfolios is still convex because any weighted average of two portfolios will also have non-negative components and still sum to 100. Therefore, the variance function still has a single minimum, and a low $T$ remains effective. \\

In Problem 4, however, the feasible set includes extra constraints. By counterexample, it can be shown that these constraints break convexity because if you take two valid portfolios and average them, the result may not be non-negative. As a result, the Metropolis algorithm can become stuck in local minima. In this case, a slightly higher temperature may be needed to allow the algorithm to escape from suboptimal stable states.


\pagebreak


\subsection*{Problem 2:}


We aim to find the minimum variance portfolio using the Metropolis algorithm, allowing both long and short positions. The only constraint is that the total investment sums to \$100:
\[
\sum_{i=1}^{30} x_i = 100
\]
The variance of the portfolio serves as our energy function:
\[
E(\mathbf{x}) = \mathbf{x}^T V \mathbf{x}
\]
where \( \mathbf{x} \in \mathbb{R}^{30} \) is the portfolio vector and \( V \in \mathbb{R}^{30 \times 30} \) is the covariance matrix of stock returns. \\

To explore the state space, we define neighboring portfolios by selecting two distinct stocks $i \neq j$ and shifting \$0.01 from one to another. This maintains the total investment and yields $30 \times 29$ unique neighbors per state. \\

We ran the Metropolis algorithm at various temperatures and found that the optimal temperature, \( T = 0 \) yielded the lowest variance. The best portfolio had:
\[
\text{Var}(\mathbf{x}) = 1.472188
\]
The allocation (in dollars) is:

\begin{center}
\begin{tabular}{ll|ll}
\toprule
\textbf{Ticker} & \textbf{Allocation} & \textbf{Ticker} & \textbf{Allocation} \\
\midrule
AMGN & -0.20 & AMZN & 33.28 \\
APPL & -9.18 & AXP & -2.66 \\
BA & -2.26 & CAT & -7.73 \\
CRM & 1.65 & CSCO & -8.21 \\
CVX & 1.92 & DIS & 2.69 \\
GS & -3.14 & HD & 0.70 \\
HON & 18.12 & IBM & -13.88 \\
JNJ & 19.64 & JPM & -1.76 \\
KO & 2.06 & MCD & 17.23 \\
MMM & 8.29 & MRK & 10.27 \\
MSFT & -21.78 & NKE & 8.21 \\
NVDA & -5.77 & PG & -13.74 \\
SHW & 3.26 & TRV & -3.77 \\
UNH & -12.68 & V & -4.00 \\
VZ & 60.45 & WMT & 23.00 \\
\bottomrule
\end{tabular}
\end{center}

The resulting allocation (rounded to the nearest cent) included long positions in stable stocks such as Verizon and Walmart, and short positions in more volatile and highly correlated stocks such as Microsoft and Apple.

To validate the results of the Metropolis algorithm, we compared its output to the analytical solution for the minimum variance portfolio:
\[
\mathbf{x}_b = \frac{100}{c} V^{-1} \mathbf{e}, \quad \text{where } \mathbf{e} = (1,\dots,1)^T, \quad c = \mathbf{e}^T V^{-1} \mathbf{e}
\]
Both the Metropolis algorithm and the analytical solution produced a portfolio with the variance:
\[
\text{Var}(\mathbf{x}) = 1.472188
\]
This confirms that our implementation was numerically correct and precise. This demonstrates that the Metropolis algorithm can accurately recover the true minimum variance portfolio when the state space is unconstrained and continuous.

\pagebreak


\subsection*{Problem 3:}
We now seek the minimum variance portfolio under the constraint that short positions are not allowed. That is, each $x_i \geq 0$, with the total investment still equal to \$100. We apply the Metropolis algorithm with the same energy function, introducing a penalty to discourage states with short positions:
\[
E(\mathbf{x}) = 
\begin{cases}
\mathbf{x}^T V \mathbf{x}, & \text{if } x_i \geq 0 \text{ for all } i \\
1000, & \text{otherwise}
\end{cases}
\]

We ran the algorithm with \( T = 0 \), accepting only variance-reducing steps. The best portfolio found had variance: 
\[
\text{Var}(\mathbf{x}) = 2.173518
\]

Here is the allocation (rounded to the nearest cent):

\begin{center}
\begin{tabular}{ll|ll}
\toprule
\textbf{Ticker} & \textbf{Allocation} & \textbf{Ticker} & \textbf{Allocation} \\
\midrule
AMGN & 0.00 & AMZN & 15.45 \\
AAPL & 0.00 & AXP  & 0.00 \\
BA   & 0.00 & CAT  & 0.00 \\
CRM  & 0.00 & CSCO & 0.00 \\
CVX  & 0.00 & DIS  & 0.00 \\
GS   & 0.00 & HD   & 0.00 \\
HON  & 0.00 & IBM  & 0.00 \\
JNJ  & 10.00 & JPM  & 0.00 \\
KO   & 5.87 & MCD  & 3.58 \\
MMM  & 5.10 & MRK  & 10.00 \\
MSFT & 0.00 & NKE  & 0.00 \\
NVDA & 0.00 & PG   & 0.00 \\
SHW  & 0.00 & TRV  & 0.00 \\
UNH  & 0.00 & V    & 0.00 \\
VZ   & 25.00 & WMT  & 25.00 \\
\bottomrule
\end{tabular}
\end{center}

The resulting allocation is highly concentrated, with the majority of the capital allocated just to a handful of stocks. We observe that many of these stocks come from sectors that are traditionally considered more stable, such as telecommunications and healthcare. This suggests that these companies likely have both individual low variances and relatively weak correlations with other stocks, as encoded in the covariance matrix.

\pagebreak



\subsection*{Problem 4:}
In this problem, we seek the minimum variance portfolio under these constraints:
\begin{itemize}
    \item No short positions ($x_i \geq 0$)
    \item Total investment sums to \$100
    \item No stock may receive more than \$25
    \item At most three stocks may receive more than \$10
\end{itemize}

To enforce these constraints, we assign an artificial penalty of $\text{Var}(\mathbf{x}) = 1000$ to any portfolio that violates them. This discourages the algorithm from exploring infeasible proposals. We ran the algorithm at temperature $T = 0$, accepting only moves that strictly reduce variance. We found that the best portfolio had:
\[
\text{Var}(\mathbf{x}) = 2.341931
\]

Here is the allocation (rounded to the nearest cent):

\begin{center}
\begin{tabular}{ll|ll}
\toprule
\textbf{Ticker} & \textbf{Allocation} & \textbf{Ticker} & \textbf{Allocation} \\
\midrule
AMGN & 0.00 & AMZN & 15.45 \\
AAPL & 0.00 & AXP  & 0.00 \\
BA   & 0.00 & CAT  & 0.00 \\
CRM  & 0.00 & CSCO & 0.00 \\
CVX  & 0.00 & DIS  & 0.00 \\
GS   & 0.00 & HD   & 0.00 \\
HON  & 0.00 & IBM  & 0.00 \\
JNJ  & 10.00 & JPM  & 0.00 \\
KO   & 5.87 & MCD  & 3.58 \\
MMM  & 5.10 & MRK  & 10.00 \\
MSFT & 0.00 & NKE  & 0.00 \\
NVDA & 0.00 & PG   & 0.00 \\
SHW  & 0.00 & TRV  & 0.00 \\
UNH  & 0.00 & V    & 0.00 \\
VZ   & 25.00 & WMT  & 25.00 \\
\bottomrule
\end{tabular}
\end{center}




\pagebreak


\subsection*{Problem 5:}
We now consider a restricted subset of portfolios known as "silly" portfolios, defined by the following structure:
\begin{itemize}
    \item Exactly 10 stocks contribute \$6 each
    \item Another 10 stocks contribute \$4 each
    \item The remaining 10 stocks contribute \$0 each
\end{itemize}
This portfolio always sums to \$100, and there are approximately 
\[
\binom{30}{10} \times \binom{20}{10} \approx 5.55 \times 10^{12}
\]
such portfolios in total. \\

To find the minimum variance silly portfolio, we initialized a random feasible allocation of \$6 for the first ten stocks, \$4 for the next ten, and \$0 for the remaining ten. At each step, we selected one of the 20 stocks in the portfolio. We swapped its contribution with one of the stocks not in the portfolio. This ensures that the portfolio preserves its structure and sums to \$100. To determine which proposal states to accept, we experimented with a variety of values but eventually chose $T = 0.05$, ensuring that the algorithm does not get stuck at a stable state. \\

The best silly portfolio we found had:
\[
\text{Var}(\mathbf{x}) = 3.430752
\]

Here is the portfolio's allocation:
\begin{center}
\begin{tabular}{ll|ll} 
\toprule 
\textbf{Ticker} & \textbf{Allocation} & \textbf{Ticker} & \textbf{Allocation} \\
\midrule 
AMGN & 6.00 & AMZN & 6.00 \\
AAPL & 4.00 & AXP & 0.00 \\
BA & 0.00 & CAT & 4.00 \\
CRM & 4.00 & CSCO & 4.00 \\
CVX & 0.00 & DIS & 4.00 \\
GS & 0.00 & HD & 0.00 \\
HON & 4.00 & IBM & 4.00 \\
JNJ & 6.00 & JPM & 0.00 \\
KO & 6.00 & MCD & 6.00 \\
MMM & 6.00 & MRK & 6.00 \\
MSFT & 0.00 & NKE & 4.00 \\
NVDA & 0.00 & PG & 6.00 \\
SHW & 4.00 & TRV & 4.00 \\
UNH & 0.00 & V & 0.00 \\
VZ & 6.00 & WMT & 6.00 \\
\bottomrule
\end{tabular}
\end{center}


\pagebreak


\subsection*{Problem 6:}
A portfolio state \( x \) is considered a stable state if its energy \( E(x) \) is less than or equal to the energy of all neighboring states \( y \), i.e., \( E(y) > E(x) \) for all neighbors \( y \) of \( x \). In this problem, we aim to find such a stable state under the silly portfolio structure. \\

To find a stable state, we ran the Metropolis algorithm with zero temperature $T = 0$, which means only downhill moves to lower variance states were accepted. Starting from a silly portfolio initialized in the manner discussed in Problem 5, the algorithm quickly converged to a local minimum where no further improving swaps were found. This portfolio had a variance of:
\[
\text{Var}(\mathbf{x}) = 3.721078
\]
Which is clearly larger than the global minimum variance portfolio found in Problem 5. This was the allocation for this portfolio:

\begin{center}
\begin{tabular}{ll|ll} 
\toprule 
\textbf{Ticker} & \textbf{Allocation} & \textbf{Ticker} & \textbf{Allocation} \\
\midrule 
AMGN & 4.00 & AMZN & 4.00 \\
AAPL & 4.00 & AXP  & 0.00 \\
BA   & 0.00 & CAT  & 4.00 \\
CRM  & 6.00 & CSCO & 4.00 \\
CVX  & 0.00 & DIS  & 6.00 \\
GS   & 0.00 & HD   & 0.00 \\
HON  & 6.00 & IBM  & 6.00 \\
JNJ  & 6.00 & JPM  & 0.00 \\
KO   & 4.00 & MCD  & 6.00 \\
MMM  & 6.00 & MRK  & 6.00 \\
MSFT & 0.00 & NKE  & 4.00 \\
NVDA & 0.00 & PG   & 6.00 \\
SHW  & 6.00 & TRV  & 4.00 \\
UNH  & 0.00 & V    & 0.00 \\
VZ   & 4.00 & WMT  & 4.00 \\
\bottomrule
\end{tabular}
\end{center}




\pagebreak

\section*{Conclusion:}
In conclusion, we tested the Metropolis algorithm across a variety of constrained optimization problems related to the minimum variance portfolio. Starting with unconstrained portfolios, we explored Metropolis's behavior under financial constraints such as the prohibition of short-selling, diversification limits, and rigid portfolio structures like “silly” portfolios. In each case, the energy function was defined as the portfolio variance \( E(x) = x^\top V x \), with the state space, neighborhood structures, and temperatures adapted accordingly.  Notably, through zero-temperature dynamics for problem 5, we observed convergence to locally stable states and not to ground states, highlighting the existence of multiple local minima under rigid structural constraints. These results emphasize both the power and limitations of the Metropolis Algorithm in practical portfolio design.

\section*{Contributions:}
We all worked together on the coding of each problem to ensure that there were no discrepancies. We provided feedback and clarified the details throughout the process, refining our implementations until all discrepancies were resolved. Once we finalized the code, we worked on the report together, and we made sure to include sufficient detail, ensure proper formatting, and maintain clarity throughout.


\pagebreak



\end{document}
